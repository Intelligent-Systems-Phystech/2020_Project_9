\documentclass[12pt, twoside]{article}
\usepackage{jmlda}
\newcommand{\hdir}{.}
\usepackage[utf8]{inputenc}
\usepackage[english,russian]{babel}
\usepackage{graphicx}
% \usepackage[usenames]{color}
% \usepackage{colortbl}





\begin{document}

\title
    [Шаблон статьи для публикации] % краткое название; не нужно, если полное название влезает в~колонтитул
    {Распознавание текста на основе скелетного представления толстых линий и сверточных сетей}
\author
    { И.\,А.~Рейер, В.\,В.~Стрижов, М.~Потанин, Д.~Ожерелков В.\,А.~Шокоров} % основной список авторов, выводимый в оглавление

\email
    {author@site.ru; co-author@site.ru;  co-author@site.ru}
% \thanks
%     {}
\organization
    {$^1$Организация, адрес; $^2$Организация, адрес}
\abstract
    {
    	Исследуется метод распознавания растрового представления изображения символа, с помощью скелетного представления. В качестве решения используется Алгоритм Местецкого для получения графа скелетного представления, затем к нему применяется GCNN для классификации символа. 
    	
    	\textcolor{gray}{Наша модель ДОЛЖНА повысить скорость и качества обработки изображения.}
    %   В ряде экспериментов...
    
        \textcolor{gray}{Так же рассматривается задача привилегированного обучения, когда происходит совместная минимизация ошибки двух сетей, более сложной, но классической CNN и более легкой GCNN. Целью исследования является анализ изменения весов GCNN.}
        Проведены эксперименты по классификации рукописных цифр.
    	
\bigskip
\noindent
\textbf{Ключевые слова}: \emph {распознавание текста, скелетное предствавление, GCNN, привилегированное обучение}
}


%данные поля заполняются редакцией журнала
% \doi{10.21469/22233792}
% \receivedRus{01.01.2017}
% \receivedEng{January 01, 2017}

\maketitle


\section{Введение}

Мы рассматриваем изображения символа, как двумерную фигуру, ограниченной конечным множеством полигонов. Алгоритм Местецкого, основанный на прямом построении обобщенной триангуляции Делоне множества граничных элементов фигуры позволяет нам получить скелетное графовое представление данного символа.

% Цель исследования (мотивация исследования)
Цель исследования уменьшить вычислительную сложность алгоритма относительно классического подхода решения задачи.

% Предмет исследования (вокруг чего строится исследование)
Мы рассматриваем изображения символа, как двумерную фигуру, ограниченной конечным множеством полигонов, полученной из бинаризации изображения (В нашем случае под бинаризацией понимается следующее: все пиксели, цвет которых НЕ черный (то есть > 0 \textcolor{gray}{(либо можно установить threshold)} ) в нотации цветов от 0 до 255), становятся белыми)

% Исследуемая проблема (в чем заключается сложность, что требуется сделать)
Алгоритм Местецкого[], основанный на прямом построении обобщенной триангуляции Делоне множества граничных элементов фигуры позволяет нам получить скелетное графовое представление данного символа. Поэтому исследовательская часть нашей работы заключается в том, чтобы научиться качественно преобразовывать графовое представление символа в вектор, для дальнейшего решения задачи классификации. Сложность данного подхода получается из нерегулярности структуры графа, объекты, подлежащие обработке, не упорядочены и могут иметь произвольную размерность и структуру связей. Обстоятельное исследование этой проблемы и возможных подходов к её решению можно найти в статье \cite{going_beyond_Euclidean_data} %

% Методы исследования: обзор работ, посвященных проблеме, современному состоянию исследований
Кроме канонического решения задачи обработки изображения CNN, похожая проблема решается в \cite{solution_by_SVM_classifier} там используется набор контуров и граф скелетного представелния с последующим линейным классификатором SVM. Так же существует подход использующий медианное представление \cite{solution_by_Medial_Representation_GCNN}. Существует большой набор статей Липкиной А.Л. и Местецкого Л.М. посвященных решению нашей задачи, например \cite{Lipkina_Mestetskiy_1} \cite{Lipkina_Mestetskiy_2}.

% Решаемая в данной работе задача
Мы хотим найти оптимальные параметры графового представления (таких как координаты вершин, их валентность, степень, тольщина линий и тд.) описывающие символ, для их подбора можно воспользоваться например \cite{Structural_functional_analisys}.
% Предлагаемое решение
% Работа или работы описывающие наиболее близкие решения
% Анализ сильных и слабых сторон предлагаемого решения
% Цель эксперимента, на каких данных будет выполнен эксперимент



\bibliographystyle{unsrt}
\bibliography{References}






\end{document}